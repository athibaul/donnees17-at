
\documentclass[french,a4paper]{article}

\usepackage[T1]{fontenc}
\usepackage[utf8]{inputenc}
\usepackage{lmodern}
\usepackage[a4paper,margin=0.85in]{geometry} % Réduire les marges
\usepackage[francais]{babel}
\usepackage{amsmath}
\usepackage{amssymb}
\usepackage{amsthm}

\usepackage{cite} % Pour importer une bibliographie

\usepackage{bbm}
\usepackage{mathtools}
%\usepackage{ulem} % Pour barrer du texte
\usepackage{xcolor} % Pour colorer des symboles dans les équations

\usepackage{tikz} % Pour les dessins
\usepackage{enumitem} % Pour mettre des symboles au choix dans les énumérations
\usepackage{xfrac} % Pour faire des petites fractions : \sfrac{5}{7}

\usepackage{algorithm}
\usepackage{algpseudocode} % Pour écrire des algorithmes
\floatname{algorithm}{Algorithme} % ... en français


% Theorems
\theoremstyle{plain}
\newtheorem{theorem}{Théorème}
\newtheorem{proposition}{Proposition}
\newtheorem{lemma}{Lemme}
\newtheorem{corollary}{Corollaire}

\theoremstyle{definition}
\newtheorem{definition}{Définition}

\theoremstyle{remark}
\newtheorem{remark}{Remarque}
\newtheorem{example}{Exemple}


% Math operators
\newcommand{\scal}[2]{\left\langle #1 , #2 \right\rangle}
\DeclareMathOperator{\IR}{\mathbb{R}}
\DeclareMathOperator*{\argmin}{argmin}
\DeclareMathOperator*{\argmax}{argmax}
\DeclareMathOperator{\One}{\mathbbm{1}}
\DeclareMathOperator{\Ccal}{\mathcal{C}}
\DeclareMathOperator{\logsumexp}{logsumexp}
\DeclareMathOperator{\diag}{diag}
\DeclareMathOperator{\MAP}{MAP}

\newcommand{\norm}[1]{\left\lVert #1 \right\rVert}
\renewcommand{\epsilon}{\varepsilon}



\title{Mathematical foundations of data science\\
Blind deconvolution}
\date{January 2018}
\author{Alexis THIBAULT}

\begin{document}

\maketitle


\section{Introduction}
Blur is a common artifact in medical imagery, astronomy, microscopy, or even photography, which is generally undesirable as it obscures significant information.
Such degradation is often approximately linear and shift-invariant, and can thus be modeled by a convolution with a kernel.

When the \emph{point spread function} (PSF) can be measured, as is sometimes the case in microscopy, the convolution kernel is known.
Restoring the original image is then a \emph{non-blind deconvolution problem}.
This simple inverse problem can be solved efficiently.

\emph{Blind deconvolution}, which aims at restoring clearness of the image when the convolution kernel is unknown, is more challenging.
It tries to estimate both the kernel and the original image, based on the blurry one.
Since the number of unknowns exceeds the data, the problem is ill-posed; one therefore needs to introduce assumptions about natural images.

\section{Mathematical formulation}
\subsection{Problem}
Suppose we are given a distorted observation $y$ of a ground truth signal $x$, and we would like to recover x. The distortion is a convolution with an unknown kernel $k$, followed by the introduction of noise $w$.
\[
y = k * x + w.
\]
We will focus on the case when $x$ and $y$ are two-dimensional grids of pixels, but the methods can be easily generalized to higher dimensions.


\subsection{Méthode de résolution}
On choisit un modèle pour les images naturelles, donné par une fonction de densité $p_x$, un modèle pour les noyaux $p_k$, et un modèle pour le bruit $p_w$.
Étant donnés ces modèles, on effectue une estimation pour le noyau en considérant le \emph{maximum a posteriori} ($\MAP_k$)~:
\[
\hat{k} = \argmax p\left( k \mid y \right) = \argmax \int p\left( x,k \mid y \right) dx,
\]
puis on calcule la déconvolution (non aveugle) à partir de ce noyau. La probabilité conditionnelle est donnée par les formules classiques~:
\[
p\left(x,k \mid y \right) = \frac{p(x,k,y)}{p(y)} = \frac{p(y \mid x,k) \; p(x) \; p(k)}{p(y)}
\]
Le dénominateur est une constante~; le premier facteur du numérateur est relatif au modèle du bruit $w = y - k*x$, le deuxième au modèle d'images, et le troisième au modèle de noyaux. On peut donc réécrire la fonction à maximiser en faisant apparaître explicitement les modèles~:
\[
\hat{k} = \argmax \int p_x(x) p_k(k) p_w(y - k*x) dx
\]

\subsection{Modèles a priori}

Considérons les opérateurs de dérivation discrète $(f_h,f_v) = ((-1,1),(-1,1)^T)$. Le modèle qu'on prend pour les images naturelles se base sur l'idée que les dérivées sont parcimonieuses. On se donne une fonction $\rho$, mélange de gaussiennes, puis on suppose toutes les coordonnées des dérivées indépendantes de loi $\rho$.
\begin{align*}
\rho((f_h*x)_i) &= \sum_j \frac{\pi_j}{\sigma_j \sqrt{2\pi}}e^{-\frac{1}{2\sigma_j^2}\norm{(f_h*x)_i}^2} \\
p_x(x) &= \prod_i \rho\left((f_h*x)_i \right) \prod_i \rho\left((f_v*x)_i\right) .
\end{align*}
On suppose la loi du noyau uniforme. En prenant un bruit blanc gaussien de variance $\eta^2$ pour $w$, on obtient~:
\[
p(y\mid x,k) = \frac{1}{(\eta\sqrt{2\pi})^N} e^{-\frac{1}{2\eta^2}\norm{k*x-y}^2} .
\]

L'intégrale à maximiser est donc~:
\[
\hat{k} = \argmax \int \exp\left( - \frac{\norm{k*x-y}^2}{2\eta^2} +  \sum_i \log\left(\rho((f_h*x)_i)\rho((f_v*x)_i)\right)  + c \right) .
\]

La difficulté ici est que l'on effectue un calcul d'intégrale non trivial. On s'intéresse donc à des stratégies d'approximation.

\section{Algorithme EM}
Une méthode classique pour calculer un maximum a posteriori (MAP) ou un maximum de vraisemblance (ML, pour \textit{maximum likelihood}) est l'algorithme Espérance-Maximisation (en anglais \textit{expectation-maximization}, ou EM). Cet algorithme permet d'estimer le maximum a posteriori lorsque le modèle probabiliste dépend de variables non observables.

Ici on considère que l'image de départ $x$ est une variable cachée. 


\end{document}


