
\documentclass[french,a4paper]{article}

\usepackage[T1]{fontenc}
\usepackage[utf8]{inputenc}
\usepackage{lmodern}
\usepackage[a4paper,margin=0.85in]{geometry} % Réduire les marges
\usepackage[francais]{babel}
\usepackage{amsmath}
\usepackage{amssymb}
\usepackage{amsthm}

\usepackage{cite} % Pour importer une bibliographie

\usepackage{bbm}
\usepackage{mathtools}
%\usepackage{ulem} % Pour barrer du texte
\usepackage{xcolor} % Pour colorer des symboles dans les équations

\usepackage{tikz} % Pour les dessins
\usepackage{enumitem} % Pour mettre des symboles au choix dans les énumérations
\usepackage{xfrac} % Pour faire des petites fractions : \sfrac{5}{7}

\usepackage{algorithm}
\usepackage{algpseudocode} % Pour écrire des algorithmes
\floatname{algorithm}{Algorithme} % ... en français


% Theorems
\theoremstyle{plain}
\newtheorem{theorem}{Théorème}
\newtheorem{proposition}{Proposition}
\newtheorem{lemma}{Lemme}
\newtheorem{corollary}{Corollaire}

\theoremstyle{definition}
\newtheorem{definition}{Définition}

\theoremstyle{remark}
\newtheorem{remark}{Remarque}
\newtheorem{example}{Exemple}


% Math operators
\newcommand{\scal}[2]{\left\langle #1 , #2 \right\rangle}
\DeclareMathOperator{\IR}{\mathbb{R}}
\DeclareMathOperator*{\argmin}{argmin}
\DeclareMathOperator*{\argmax}{argmax}
\DeclareMathOperator{\One}{\mathbbm{1}}
\DeclareMathOperator{\Ccal}{\mathcal{C}}
\DeclareMathOperator{\logsumexp}{logsumexp}
\DeclareMathOperator{\diag}{diag}
\DeclareMathOperator{\MAP}{MAP}

\newcommand{\norm}[1]{\left\lVert #1 \right\rVert}
\renewcommand{\epsilon}{\varepsilon}



\title{Outils mathématiques pour la science des données\\
Déconvolution aveugle}
\date{Janvier 2018}
\author{Alexis THIBAULT}

\begin{document}

\maketitle


\section{Introduction}
La déconvolution aveugle (\textit{blind deconvolution} en anglais) est une technique de traitement d'image visant à restaurer une image floutée, où le noyau de convolution (PSF, pour \textit{point spread function}) est inconnu. En faisant des hypothèses sur les images de départ et sur les noyaux, on parvient à estimer le noyau et à restituer une image plus nette. Les applications principales sont en astronomie et en imagerie médicale.

\section{Formulation mathématique}
\subsection{Problème}
On dispose d'une observation déformée et bruitée $y$ d'un signal d'origine $x$ qu'on souhaite récupérer. L'altération est modélisée par une convolution par un noyau $k$ et l'ajout d'un bruit $w$.
\[
y = k * x + w.
\]
Ici on s'intéressera principalement au cas où $x$ et $y$ sont des images à deux dimensions (donc des grilles de pixels).
\subsection{Méthode de résolution}
On choisit un modèle pour les images naturelles, donné par une fonction de densité $p_x$, un modèle pour les noyaux $p_k$, et un modèle pour le bruit $p_w$.
Étant donnés ces modèles, on effectue une estimation pour le noyau en considérant le \emph{maximum a posteriori} ($\MAP_k$)~:
\[
\hat{k} = \argmax p\left( k \mid y \right) = \argmax \int p\left( x,k \mid y \right) dx,
\]
puis on calcule la déconvolution (non aveugle) à partir de ce noyau. La probabilité conditionnelle est donnée par les formules classiques~:
\[
p\left(x,k \mid y \right) = \frac{p(x,k,y)}{p(y)} = \frac{p(y \mid x,k) \; p(x) \; p(k)}{p(y)}
\]
Le dénominateur est une constante~; le premier facteur du numérateur est relatif au modèle du bruit $w = y - k*x$, le deuxième au modèle d'images, et le troisième au modèle de noyaux. On peut donc réécrire la fonction à maximiser en faisant apparaître explicitement les modèles~:
\[
\hat{k} = \argmax \int p_x(x) p_k(k) p_w(y - k*x) dx
\]

\subsection{Modèles a priori}

Considérons les opérateurs de dérivation discrète $(f_h,f_v) = ((-1,1),(-1,1)^T)$. Le modèle qu'on prend pour les images naturelles se base sur l'idée que les dérivées sont parcimonieuses. On se donne une fonction $\rho$, mélange de gaussiennes, puis on suppose toutes les coordonnées des dérivées indépendantes de loi $\rho$.
\begin{align*}
\rho((f_h*x)_i) &= \sum_j \frac{\pi_j}{\sigma_j \sqrt{2\pi}}e^{-\frac{1}{2\sigma_j^2}\norm{(f_h*x)_i}^2} \\
p_x(x) &= \prod_i \rho\left((f_h*x)_i \right) \prod_i \rho\left((f_v*x)_i\right) .
\end{align*}
On suppose la loi du noyau uniforme. En prenant un bruit blanc gaussien de variance $\eta^2$ pour $w$, on obtient~:
\[
p(y\mid x,k) = \frac{1}{(\eta\sqrt{2\pi})^N} e^{-\frac{1}{2\eta^2}\norm{k*x-y}^2} .
\]

L'intégrale à maximiser est donc~:
\[
\hat{k} = \argmax \int \exp\left( - \frac{\norm{k*x-y}^2}{2\eta^2} +  \sum_i \log\left(\rho((f_h*x)_i)\rho((f_v*x)_i)\right)  + c \right) .
\]

La difficulté ici est que l'on effectue un calcul d'intégrale non trivial. On s'intéresse donc à des stratégies d'approximation.

\section{Algorithme EM}
Une méthode classique pour calculer un maximum a posteriori (MAP) ou un maximum de vraisemblance (ML, pour \textit{maximum likelihood}) est l'algorithme Espérance-Maximisation (en anglais \textit{expectation-maximization}, ou EM). Cet algorithme permet d'estimer le maximum a posteriori lorsque le modèle probabiliste dépend de variables non observables.

Ici on considère que l'image de départ $x$ est une variable cachée. 


\end{document}


